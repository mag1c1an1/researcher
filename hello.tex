\documentclass{article}
\usepackage[UTF8]{ctex}
\usepackage{amsmath}
\begin{document}
对于 $\forall$ x, y $\in$ C 与 $\forall$  $\lambda\in \lbrack 0, 1 \rbrack $ ,有
$\lambda x + (1-\lambda)y \in C$


由于全平面是一个凸集,故任何平面点集都可用全平面盖住,即能被凸集盖住,从而盖住该凸集的所有凸集的交集存在,即凸包存在. \\

而如果某个凸集A有两个凸包M1与M2,则$M1 \bigcap M2$也能盖住凸集A,且$M1 \bigcap M2 \subset M1$,但M1是A的凸包,故$ M1 \subset M1 \bigcap M2$,故$M1 \bigcap M2 = M1$.同理$M1 \bigcap M2 = M2$.即$M1 = M2$


$$
\left|
\begin{matrix}
x_1 & y_1 & 1 \\
x_2 & y_2 & 1 \\
x_3 & y_3 & 1 \\
\end{matrix}
\right| = x_1y_2+x_3y_1+x_2y_3-x_3y_2-x_2y_1-x_1y_3
$$

$\mathcal{O}(n\log{}n)$
\emph{mx}

It is worth noting that there is no simple linear relation between d and r with performances measured by mIoU, though d = 8r is a satisfactory choice. Experiments show that even r = 8 performs well, revealing that it can be very cheap for modeling the global context

\end{document}